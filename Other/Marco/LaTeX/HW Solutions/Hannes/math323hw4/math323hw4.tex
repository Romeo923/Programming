% This is a simple sample document.  For more complicated documents take a look in the excersice tab. Note that everything that comes after a % symbol is treated as comment and ignored when the code is compiled.

\documentclass{article} % \documentclass{} is the first command in any LaTeX code.  It is used to define what kind of document you are creating such as an article or a book, and begins the document preamble

\usepackage{amsmath} % \usepackage is a command that allows you to add functionality to your LaTeX code
\usepackage{mathtools}

\DeclarePairedDelimiter{\opair}{\langle}{\rangle}

\title{Math323} % Sets article title
\author{Homework 4} % Sets authors name
\date{\today} % Sets date for date compiled

% The preamble ends with the command \begin{document}
\begin{document} % All begin commands must be paired with an end command somewhere
    \maketitle % creates title using infromation in preamble (title, author, date)
    
    \section*{Problem 3} % creates a section
    \mbox{$Show\: that\: $}\vspace{0.5in}\\
    \textbf{Proof:}
    \begin{equation} % Creates an equation environment and is compiled as math
      \frac{1}{N}\sum_{i=0}^{N-1} (x_i-m_x)(y_i-m_y) =
    \end{equation}
    \begin{equation}
      \frac{1}{N}\sum_{i=0}^{N-1} x_iy_i-x_im_y-y_im_x+m_xmy =
    \end{equation}
    \begin{equation}
      \frac{1}{N}\sum_{i=0}^{N-1} x_iy_i - \frac{1}{N}\sum_{i=0}^{N-1} x_im_y - \frac{1}{N}\sum_{i=0}^{N-1} y_im_x + \frac{1}{N}\sum_{i=0}^{N-1} m_xm_y =
    \end{equation}
    \begin{equation}
      \frac{1}{N}\sum_{i=0}^{N-1} x_iy_i - \frac{m_y}{N}\sum_{i=0}^{N-1} x_i - \frac{m_x}{N}\sum_{i=0}^{N-1} y_i + \frac{1}{N}\sum_{i=0}^{N-1} m_xm_y =
    \end{equation}
    \begin{equation}
      \overline{xy} - m_y\bar{x} - m_x\bar{y} + \bar{x}\bar{y} =
    \end{equation}
    \begin{equation}
      \overline{xy} - \bar{x}\bar{y} - \bar{x}\bar{y} + \bar{x}\bar{y} =
    \end{equation}
    \begin{equation}
      \overline{xy} - \bar{x}\bar{y}
    \end{equation}
\end{document} % This is the end of the document
