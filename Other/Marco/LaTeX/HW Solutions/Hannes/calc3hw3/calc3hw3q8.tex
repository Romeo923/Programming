% This is a simple sample document.  For more complicated documents take a look in the excersice tab. Note that everything that comes after a % symbol is treated as comment and ignored when the code is compiled.

\documentclass{article} % \documentclass{} is the first command in any LaTeX code.  It is used to define what kind of document you are creating such as an article or a book, and begins the document preamble

\usepackage{amsmath} % \usepackage is a command that allows you to add functionality to your LaTeX code
\usepackage{mathtools}

\DeclarePairedDelimiter{\opair}{\langle}{\rangle}

\title{Math215} % Sets article title
\author{Homework 3} % Sets authors name
\date{\today} % Sets date for date compiled

% The preamble ends with the command \begin{document}
\begin{document} % All begin commands must be paired with an end command somewhere
    \maketitle % creates title using infromation in preamble (title, author, date)
    
    \section*{Problem 8} % creates a section
    \mbox{$Show\: that\: \vec{A}\cdot(\vec{A}\times\vec{B})=\vec{0}\: and\: \vec{B}\cdot(\vec{A}\times\vec{B})=\vec{0}\: for\: all\: \vec{A}=\opair{a_1,a_2,a_3}\: and\: \vec{B}=\opair{b_1,b_2,b_3}$}\vspace{0.5in}\\
    \textbf{Proof:}
    \begin{equation} % Creates an equation environment and is compiled as math
      \vec{A}\cdot(\vec{A}\times\vec{B})=
    \end{equation}
    \begin{equation}
      \opair{a_1,a_2,a_3}\cdot(\opair{a_1,a_2,a_3}\times\opair{b_1,b_2,b_3})= 
    \end{equation}
    \begin{equation}
      \opair{a_1,a_2,a_3}\cdot(\opair{a_2b_3-a_3b_2,a_3b_1-a_1b_3,a_1b_2-a_2b_1})=
    \end{equation}
    \begin{equation}
      a_1(a_2b_3-a_3b_2)+a_2(a_3b_1-a_1b_3)+a_3(a_1b_2-a_2b_1)=
    \end{equation}
    \begin{equation}
      a_1a_2b_3-a_1a_3b_2+a_2a_3b_1-a_2a_1b_3+a_3a_1b_2-a_3a_2b_1=
    \end{equation}
    \begin{equation}
      a_1a_2b_3-a_1a_2b_3+a_2a_3b_1-a_2a_3b_1+a_1a_3b_2-a_1a_3b_2=
    \end{equation}
    \begin{equation}
      0
    \end{equation}

    
    \begin{equation}
      \vec{B}\cdot(\vec{A}\times\vec{B})=
    \end{equation}
    \begin{equation}
      \opair{b_1,b_2,b_3}\cdot(\opair{a_1,a_2,a_3}\times\opair{b_1,b_2,b_3})=
    \end{equation}
    \begin{equation}
      \opair{b_1,b_2,b_3}\cdot(\opair{a_2b_3-a_3b_2,a_3b_1-a_1b_3,a_1b_2-a_2b_1})=
    \end{equation}
    \begin{equation}
      b_1(a_2b_3-a_3b_2)+b_2(a_3b_1-a_1b_3)+b_3(a_1b_2-a_2b_1)=
    \end{equation}
    \begin{equation}
      b_1a_2b_3-b_1a_3b_2+b_2a_3b_1-b_2a_1b_3+b_3a_1b_2-b_3a_2b_1=
    \end{equation}
    \begin{equation}
      b_1a_2b_3-b_1a_2b_3+b_2a_3b_1-b_2a_3b_1+b_3a_1b_2-b_3a_1b_2=
    \end{equation}
    \begin{equation}
      0
    \end{equation}
    
\end{document} % This is the end of the document
