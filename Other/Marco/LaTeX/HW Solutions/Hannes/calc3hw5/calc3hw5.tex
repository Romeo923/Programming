\documentclass{article} % \documentclass{} is the first command in any LaTeX code.  It is used to define what kind of document 

\usepackage{amsmath} % \usepackage is a command that allows you to add functionality to your LaTeX code            
\usepackage{mathtools}
\usepackage{amssymb}
\usepackage[margin=0.75in,footskip=0.25in]{geometry}

\title{Math215} % Sets article title

\author{Homework 5, Problem 2} % Sets authors name
\date{\today} 

\begin{document} % All begin commands must be paired with an end command somewhere                    
\maketitle % creates title using infromation in preamble (title, author, date)
Taylor series centered at a:\\ $$f(x) \approx \sum_{n=0}^{\infty}\frac{f^{(n)}(a)}{n!}(x-a)^{n}$$
\section*{8.7 Problem 14} % creates a section
\begin{displaymath}
  \begin{array}{|c|c|c|c|c}
    f(x)=\frac{1}{x} & f^{'}(x)=\frac{-1}{x^{2}} & f^{''}(x)=\frac{2}{x^{3}} & f^{'''}(x)=\frac{-6}{x^{4}} & \cdot\cdot\cdot\\
    \\
    \hline
    \\
    f(-3)=\frac{-1}{3} & f^{'}(-3)=\frac{-1}{9} & f^{''}(-3)=\frac{-2}{27} & f^{'''}(-3)=\frac{-2}{27} & \cdot\cdot\cdot\\
  \end{array}
\end{displaymath}
\\
$$f(-3) = \frac{-1}{3} + \frac{-1}{9}\frac{1}{1!}(x+3) + \frac{-2}{27}\frac{1}{2!}(x+3)^{2} + \frac{-2}{27}\frac{1}{3!}(x+3)^{3} + ...$$\\
$$= \frac{-1}{3}\Big{(}1 + \frac{1}{3}(x+3) + \frac{1}{9}(x+3)^{2} + \frac{1}{27}(x+3)^{3} + ...\Big{)}$$\\
$$= \frac{-1}{3}\sum_{n=0}^{\infty}\frac{(x+3)^{n}}{3^{n}}$$
\pagebreak
\section*{8.7 Problem 15}
\begin{displaymath}
  \begin{array}{|c|c|c|c|c}
    f(x)=e^{2x} & f^{'}(x)=2e^{2x} & f^{''}(x)=4e^{2x} & f^{'''}(x)=8e^{2x} & \cdot\cdot\cdot\\
    \\
    \hline
    \\
    f(3)=e^{6} & f^{'}(3)=2e^{6} & f^{''}(3)=4e^{6} & f^{'''}(3)=8e^{6} & \cdot\cdot\cdot\\
  \end{array}
\end{displaymath}  
\\
$$f(3)=e^{6} + 2e^{6}(x-a) + \frac{4e^{6}}{2!}(x-a)^{2} + \frac{8e^{6}}{3!}(x-a)^{3} + ...$$\\
$$= e^{6}\sum_{n=0}^{\infty} \frac{2^{n}}{n!}(x-3)^{n}$$

\section*{8.7 Problem 16}
\begin{displaymath}
  \begin{array}{|c|c|c|c|c|c}
    f(x)=sin(x) & f^{'}(x)=cos(x) & f^{''}(x)=-sin(x) & f^{'''}(x)=-cos(x) & f^{(4)}(x)=sin(x) & \cdot\cdot\cdot\\
    \\
    \hline
    \\
    f(\frac{\pi}{2})=sin(\frac{\pi}{2})=1 & f^{'}(\frac{\pi}{2})=cos(\frac{\pi}{2})=0 & f^{''}(\frac{\pi}{2})=-sin(\frac{\pi}{2})=-1 & f^{'''}(\frac{\pi}{2})=-cos(\frac{\pi}{2})=0 & f^{(4)}(\frac{\pi}{2})=sin(\frac{\pi}{2})=1 & \cdot\cdot\cdot\\
  \end{array}
\end{displaymath}
\\
$$f(\frac{\pi}{2}) = sin(\frac{\pi}{2}) + \cos(\frac{\pi}{2})(x-a) + \frac{-sin(\frac{\pi}{2})}{2!}(x-a)^{2} + \frac{-cos(\frac{\pi}{2})}{3!}(x-a)^{3} + \frac{sin(\frac{\pi}{2})}{4!}(x-\frac{\pi}{2})^{4} + ...$$\\
$$= 1 + 0 - \frac{(x-\frac{\pi}{2})^{2}}{2!} + 0 + \frac{(x-\frac{\pi}{2})^{4}}{4!} + ...$$\\
$$= \sum_{n=0}^{\infty} \frac{(-1)^{n}}{(2n)!}\Big{(}x-\frac{\pi}{2}\Big{)}^{2n}$$

\end{document}
