\documentclass{article}
\usepackage{amsmath, amssymb}
\usepackage{hyperref}
\usepackage[top=2cm,left=2cm,right=2cm,bottom=2cm]{geometry}

\begin{document}
    \begin{center}
        Romeo Capozziello HW 1
    \end{center}
    Problem 1: \\ 
    Using direct proof, prove: If n is any even interger, then $(-1)^n$ = 1\\
    \\
    If n is even then n = 2k\\
    Then $(-1^n)$ = $(-1^{2k})$ = $(-1^k) * (-1^k)$\\
    If $(-1^k) = 1$, then $(-1^k) * (-1^k) = 1 * 1 = 1$ \\
    If $(-1^k) = -1$, then $(-1^k) * (-1^k) = -1 * -1 = 1$ \\
    Then $(-1^k) * (-1^k) = 1 \rightarrow (-1^{2k}) = 1 \rightarrow (-1^n) = 1$\\
    \\
    \rule{\textwidth}{0.5pt}\\
    Problem 2: \\
    Using induction proof, prove for integer $n \geq 5$, $4n < 2^n$\\
    \\
    Base Case: n = 5, $4(5) < 2^5 \rightarrow 20 < 32$\\
    Assume true for n = k, $4k < 2^k$\\
    $4(k+1) = 4k + 4)$\\
    $2^{k+1} = 2^k*2 = 2^k + 2^k$\\\\
    $4k + 4 < 2^k + 2k$\\
    $4k < 2^k + 2^k - 4$\\
    $4k < 2^k$ by assumption\\
    Then $4k < 2^k < 2^k + 2^k - 4$\\
    Then $4k < 2^k + 2^k - 4 \rightarrow 4k < 2^{k+1} - 4 \rightarrow 4k + 4 < 2^{k+1}$\\
    Then $4(k+1) < 2^{k+1}$\\
    \\
    \rule{\textwidth}{0.5pt}\\
    Problem 3: \\
    Prove by induction that $(11^n - 6)$ is divisible by 5 for every possible integer n.\\
    \\
    Base Case: n = 0, 5 $|$ $(11^0 - 6) \rightarrow$ 5 $|$ (1-6) $\rightarrow$ 5 $|$ -5\\
    Assume true for n = k, 5 $|$ $(11^k - 6)$\\\\
    5 $|$ $(11^{k+1} - 6)$\\
    $11^{k+1} - 6 = 11*11^k - 6$\\
    5 $|$ $(11^k - 6)$ by assumption, thus $11^k = 5x + 6$\\
    $11^{k+1} - 6$  =  $11*11^k - 6$  =  $11 * (5x + 6) - 6$  =  $55x + 66 - 6$  =  $55x + 60$\\
    $55x + 60 = 5*(11x + 12)$, 5 $|$ $5(11x + 12)$\\
    5 $|$ $(11^{k+1} - 6)$\\
    \\
    \rule{\textwidth}{0.5pt}\\
    Problem 4: \\
    Prove the following statement by Contradiction\\
    The sum of a rational number and an irrational number is irrational.\\
    \\
    Assume a the sum of a rational and irrational number was rational\\
    Let x be an irrational number\\\\
    Then $\frac{a}{b} + x = \frac{c}{d}$\\
    Then x = $\frac{c}{d} - \frac{a}{b}$\\
    Then x = $\frac{bc - da}{bd}$\\\\
    Since $\frac{a}{b}$ and $\frac{c}{d}$ are rational, then their difference, $\frac{bc - da}{bd}$ is also rational\\
    And since x = $\frac{bc - da}{bd}$, then x is rational\\
    But x is irrational by definition
\end{document}