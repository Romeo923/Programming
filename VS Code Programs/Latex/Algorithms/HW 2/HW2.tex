\documentclass{article}
\usepackage{amsmath, amssymb}
\usepackage{hyperref}
\usepackage[top=2cm,left=2cm,right=2cm,bottom=2cm]{geometry}

\begin{document}
    \begin{center}
        Romeo Capozziello HW 2
    \end{center}
    \underline{Problem 1:}\\\\
    A binary search of a sorted array with 3 unique elements where the search is always successful can be viewed as an array with each element having a probability of being selected = 1/3 and the number of comparisonsa binary search requires to find the element is shown inside the array.
    \begin{table}[h]
        \begin{tabular}{| c | c | c |} \hline
            2 & 1 & 2 \\ \hline    
        \end{tabular}
    \end{table}
    \\
    Then the averagenumber of comparisons would be  A(3)  =  (1/3)[2 + 1 + 2] = 5/3\\
    Find the A(7) for an array with 7 elements for the same binary search.\\
    \\
    (a) Draw the 7-element array showing the number of comparisons needed to find each element.
    \begin{table}[h]
        \begin{tabular}{| c | c | c | c | c | c | c |} \hline
            3 & 2 & 3 & 1 & 3 & 2 & 3 \\ \hline    
        \end{tabular}
    \end{table}
    \\
    (b) Determine A(7)\\
    \\
    \indent A(7) = (1/7)[3 + 2 + 3 + 1 + 3 + 2 + 3] = (1/7)(17) = $\frac{17}{7}$ = 2$\frac{3}{7}$\\
    \\
    (c) Using what you learned from parts (a) and (b), find the summation expression for A(n). To simplify your work, assume that n = $2^k$–1. Use k in your summation. Leave your answer in closed form in terms of n. \\
    \\
    \indent A(n) = $\frac{\sum_{i = 1}^{k} i2^{i-1}}{2^i-1}$ = $\frac{(k-1)2^i+1}{2^i-1}$ = $\frac{(lg(n+1)-1)(n+1)+1}{n}$\\
    \\
    (d) Simplify your a(n) for large values of n.\\
    \\
    \indent $\frac{(lg(n+1)-1)(n+1)+1}{n}$ $\approx$ lg(n)-1 when n is large\\
    \rule{\textwidth}{0.5pt}\\
    \underline{Problem 2:}\\\\
    Consider the algorithm below\\Precondition: n is a non-negative integer\\
    \\
    function f(n)\\
    \{\\
    \indent temp = 0\\
    \indent if (n != 0)\\
    \indent \{\\
    \indent \indent for (i = 1; i <= 3; i++)\\
    \indent \indent \indent temp = temp + n * f(n-1)\\
    \indent \indent return temp\\
    \indent \}\\
    \indent else\\
    \indent \indent return 1;\\
    \}\\
    \\
    Solve for the closed form by repeated substitution\\
    \\
    \indent T(0) = 1\\
    \indent T(n) = 3nT(n-1)\\
    \indent T(n-1) = 3(n-1)T(n-2)\\
    \indent T(n-2) = 3(n-2)T(n-3)\\
    \indent T(n) = 3nT(n-1)\\
    \indent\hspace{0.73cm} = 3n[3(n-1)T(n-2)]\\
    \indent\hspace{0.73cm} = 3n[3(n-1)[3(n-2)T(n-3)]]\\
    \indent\hspace{0.73cm} = $3^3$n(n-1)(n-2)T(n-3)\\
    \\
    \indent T(n) = $3^n$(n-1)(n-2) ... (n-(k-2))(n-(k-1))T(n-k)\\
    \\
    \rule{\textwidth}{0.5pt}\\
    \underline{Problem 3:}\\\\
    The recurrence relation is given as \\
    \indent $a_n = 2a_{n-1} + 3a_{n-2}$\\
    Use the method linear homogeneous characteristic roots to solve for the closed form, with given initial conditions\\
    \indent $a_0 = 2$ and $a_1 = 4$\\
    \\
    (a) Find the general solution\\
    \\
    \indent $a_n$ = $r_n$\\
    \indent $r^n = 2r^{n-1} + 3r^{n-2}$\\
    \indent $r^3 = 2r + 3$\\
    \indent $r^3 - 2r - 3 = 0$\\
    \indent (r - 3)(r + 1) = 0\\
    \\
    (b) Find the Specific solution\\
    \\
    \indent Answer\\
    \\
    (c) Use your closed form result to find $a_s$\\
    \\
    \indent Answer\\
    \\
    \rule{\textwidth}{0.5pt}\\
    \underline{Problem 4:}\\\\
    Two average complexities are given below.\\Variable p is probability value in the interval of [0.0, 1.0]. Variable n is the problem size. \\
    \\
    $A_1 = p(n^2 + 1) + (1 + p)3n$\\
    $A_2 = (1 - p) (3n^2 + n) + p(2n)$\\
    \\
    (a) Determine the condition for which $A_1$ is faster than $A_2$.\\
    \\
    \indent Answer\\
    \\
    (b) Approximate the range of p values for which $A_1$ is faster than $A_2$ for large values of n.\\
    \\
    \indent Answer\\
    \\
    (c) Given n=5,find the range of values for p where $A_2$ is faster than $A_1$.\\
    \\
    \indent Answer\\
    \\
    (d) Can you determine the problem sizen where $A_1$ is always faster? Why or why not?\\
    \\
    \indent Answer\\
    \\
\end{document}