\documentclass{article}
\usepackage{amsmath, amssymb}
\usepackage{hyperref}
\usepackage[top=2cm,left=2cm,right=2cm,bottom=2cm]{geometry}

\begin{document}
    \begin{center}
        Romeo Capozziello HW 2
    \end{center}
    \underline{Problem 1:}\\\\
    A binary search of a sorted array with 3 unique elements where the search is always successful can be viewed as an array with each element having a probability of being selected = 1/3 and the number of comparisonsa binary search requires to find the element is shown inside the array.
    \begin{table}[h]
        \begin{tabular}{| c | c | c |} \hline
            2 & 1 & 2 \\ \hline    
        \end{tabular}
    \end{table}
    \\
    Then the averagenumber of comparisons would be  A(3)  =  (1/3)[2 + 1 + 2] = 5/3\\
    Find the A(7) for an array with 7 elements for the same binary search.\\
    \\
    (a) Draw the 7-element array showing the number of comparisons needed to find each element.
    \begin{table}[h]
        \begin{tabular}{| c | c | c | c | c | c | c |} \hline
            4 & 3 & 2 & 1 & 2 & 3 & 4 \\ \hline    
        \end{tabular}
    \end{table}
    \\
    (b) Determine A(7)\\
    \\
    \indent A(7) = (1/7)[4 + 3 + 2 + 1 + 2 + 3 + 4] = (1/7)(19) = 19/7\\
    \\
    (c) Using what you learned from parts (a) and (b), find the summation expression for A(n). To simplify your work, assume that n = $2^k$–1. Use k in your summation. Leave your answer in closed form in terms of n. \\
    \\
    \indent n = $2^k$ - 1, A(n) = $\frac{1}{n}(1 + \sum_{i = 2}^{k+1}$ 2i)\\\indent Ex: n = 7, 7 = $2^3$ - 1, k = 3, A(7) = $\frac{1}{7}(1 + \sum_{i = 2}^{3+1}$ 2i) = 19/7\\
    \\
    (d) Simplify your a(n) for large values of n.\\
    \\
    \indent $\sum_{i = 1}^{k} i = \frac{k(k+1)}{2} \rightarrow \sum_{i = 2}^{k} i = \frac{k(k+1)}{2} - 1 \rightarrow \sum_{i = 2}^{k+1}$ i $= \frac{(k+1)(k+2)}{2} - 1 \\\\\indent \rightarrow \sum_{i = 2}^{k+1}$ 2i $= \frac{2(k+1)(k+2)}{2} - 1 = (k+1)(k+2) - 1$\\
    \\
    \indent A(n) = $\frac{1}{n}(n(n+1) - 1) = (n+1) - \frac{1}{n}$\\
    \rule{\textwidth}{0.5pt}\\
    \underline{Problem 2:}\\\\
    Consider the algorithm below\\Precondition: n is a non-negative integer\\
    \\
    function f(n)\\
    \{\\
    \indent temp = 0\\
    \indent if (n != 0)\\
    \indent \{\\
    \indent \indent for (i = 1; i <= 3; i++)\\
    \indent \indent \indent temp = temp + n * f(n-1)\\
    \indent \indent return temp\\
    \indent \}\\
    \indent else\\
    \indent \indent return 1;\\
    \}\\
    \\
    Solvefor the closed form by repeated substitution\\
    \\
    \indent Answer
    \\
    \rule{\textwidth}{0.5pt}\\
    \underline{Problem 3:}\\\\
    The recurrence relation is given as \\
    \indent $a_n = 2a_{n-1} + 3a_{n-2}$\\
    Use the method linear homogeneous characteristic roots to solve for the closed form, with given initial conditions\\
    \indent $a_0 = 2$ and $a_1 = 4$\\
    \\
    (a) Find the general solution\\
    \\
    \indent Answer\\
    \\
    (b) Find the Specific solution\\
    \\
    \indent Answer\\
    \\
    (c) Use your closed form result to find $a_s$\\
    \\
    \indent Answer\\
    \\
    \rule{\textwidth}{0.5pt}\\
    \underline{Problem 4:}\\\\
    Two average complexities are given below.\\Variable p is probability value in the interval of [0.0, 1.0]. Variable n is the problem size. \\
    \\
    $A_1 = p(n^2 + 1) + (1 + p)3n$\\
    $A_2 = (1 - p) (3n^2 + n) + p(2n)$\\
    \\
    (a) Determine the condition for which $A_1$ is faster than $A_2$.\\
    \\
    \indent Answer\\
    \\
    (b) Approximate the range of p values for which $A_1$ is faster than $A_2$ for large values of n.\\
    \\
    \indent Answer\\
    \\
    (c) Given n=5,find the range of values for p where $A_2$ is faster than $A_1$.\\
    \\
    \indent Answer\\
    \\
    (d) Can you determine the problem sizen where $A_1$ is always faster? Why or why not?\\
    \\
    \indent Answer\\
    \\
\end{document}